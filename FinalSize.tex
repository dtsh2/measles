\documentclass[11pt,a4paper]{article}
\usepackage{amssymb,amsmath, graphicx, caption2}
\usepackage{fancyhdr}
\setlength{\textheight}{20cm}
\setlength{\topmargin}{1.5cm}
\setlength{\textwidth}{13cm}
\addtolength{\hoffset}{-1cm}
\setlength{\parindent}{0pt}
\setlength{\parskip}{10pt}
% The next 3 commands force floats to the pagetop
\makeatletter
\setlength{\@fptop}{0pt}
\makeatother
%\pagestyle {fancyplain}
\pagestyle {myheadings}
\markright{Measles vaccination by DHB}
\newcommand{\vecA}{\mathbf{A}}
\newcommand{\vecB}{\mathbf{B}}
\newcommand{\vecC}{\mathbf{C}}
\newcommand{\vecD}{\mathbf{D}}
\newcommand{\vecE}{\mathbf{E}}
\newcommand{\vecF}{\mathbf{F}}
\newcommand{\vecH}{\mathbf{H}}
\newcommand{\vecI}{\mathbf{I}}
\newcommand{\vecJ}{\mathbf{J}}
\newcommand{\vecK}{\mathbf{K}}
\newcommand{\vecM}{\mathbf{M}}
\newcommand{\vecN}{\mathbf{N}}
\newcommand{\vecP}{\mathbf{P}}
\newcommand{\vecQ}{\mathbf{Q}}
\newcommand{\vecS}{\mathbf{S}}
\newcommand{\vecT}{\mathbf{T}}
\newcommand{\vecW}{\mathbf{W}}
\newcommand{\vecZ}{\mathbf{Z}}
\newcommand{\vecb}{\mathbf{b}}
\newcommand{\vecc}{\mathbf{c}}
\newcommand{\vece}{\mathbf{e}}
\newcommand{\vecf}{\mathbf{f}}
\newcommand{\vecg}{\mathbf{g}}
\newcommand{\vech}{\mathbf{h}}
\newcommand{\veci}{\mathbf{i}}
\newcommand{\vecr}{\mathbf{r}}
\newcommand{\vecu}{\mathbf{u}}
\newcommand{\vecv}{\mathbf{v}}
\newcommand{\vecw}{\mathbf{w}}
\newcommand{\vecx}{\mathbf{x}}
\newcommand{\vecy}{\mathbf{y}}
\newcommand{\vecz}{\mathbf{z}}
\newcommand{\Ach}{\mathcal{H}}
\newcommand{\Ell}{\mathcal{L}}
\newcommand{\En}{\mathcal{N}}
\newcommand{\Pe}{\mathcal{P}}
\newcommand{\Ro}{\mathcal{R}_0}
\newcommand{\Rc}{\mathcal{R}_c}
\newcommand{\Rr}{\mathcal{R}}
\newcommand{\Qu}{\mathcal{Q}}
\newcommand{\Te}{\mathcal{T}}
\newcommand{\Ve}{\mathcal{V}}
\newcommand{\Dub}{\mathcal{W}}
\newcommand{\Zed}{\mathcal{Z}}
\newcommand{\Exp}{\mathbb{E}}
\newcommand{\de}{\mbox{d}}
\newcommand{\wrt}{\,\mbox{d}}
\newcommand{\veczero}{\mathbf{0}}
\newcommand{\vecone}{\mathbf{1}}
\newcommand{\vecalpha}{\mbox{\boldmath$\alpha$}}
\newcommand{\vectheta}{\mbox{\boldmath$\theta$}}
\newcommand{\vecphi}{\mbox{\boldmath$\phi$}}
\newcommand{\vecPhi}{\mbox{\boldmath$\Phi$}}
\newcommand{\vecSigma}{\mbox{\boldmath$\Sigma$}}
\newcommand{\vecimath}{\mbox{\boldmath$\imath$}}
\newcommand{\cosech}{\mbox{cosech}\,}
\newcommand{\sign}{\mbox{sign}\,}
\newcommand{\trace}{\mbox{trace}\,}
\newcommand{\Inf}{\mbox{Inf}\,}
\rhead{\thepage}
\cfoot{}
\renewcommand{\baselinestretch}{1.5}
\begin{document}

\title{A review of measles vaccination status in New Zealand District Health Boards\footnote{FinalSize.tex \today}}
\author{M.G. Roberts \\\\
{\small Institute of Natural \& Mathematical Sciences,}\\
{\small   New Zealand Institute for Advanced Study and the Infectious Disease Research Centre,}\\
{\small   Massey University, Private Bag 102 904, North Shore Mail Centre, Auckland, New Zealand. } \\ 
Email: \texttt{m.g.roberts@massey.ac.nz}}
\date{}
\maketitle

\thispagestyle{empty}
 
\begin{abstract}

No abstract yet. 

\end{abstract}

\newpage


%%%%%%%%%%%%%%%%%%%%%%%%%%%%%%%%%%%%%%%%%%%%%%%%%%%%%%%%%%%%%%%%%%%%%%%%%%%%%%%%%%%%%%%
\section{Introduction}

The well-known equation for the final size of an epidemic in a homogeneously mixing susceptible population is \cite{DHB}
$$\log\left(1-\Pe\right)+\Ro\Pe=0$$
where $\Ro$ is the basic reproduction number and $\Pe$ is the proportion of the population infected over the course of the outbreak.
If a proportion $x_0$ of the population is susceptible following vaccination, then the  reproduction number under vaccination is $\Rr_V=x_0\Ro$, and the final size equation becomes
$$\log\left(1-\frac{\Pe}{x_0}\right)+\Ro\Pe=0$$
Hence the relationship between the proportion initially susceptible and the proportion infected in an epidemic is
$$x_0=\frac{\Pe}{1-e^{-\Ro\Pe}}$$
In order to prevent future epidemics, it is necessary that $\Rr_V<1$. Hence, the proportion of the population that must be vaccinated to prevent future outbreaks is $x_0-1/\Ro$.

These formulae were applied at a District Health Board (DHB) level, assuming no mixing between DHBs.

\begin{table}[htdp]

\begin{center}

\begin{tabular}{lrrrr}
\hline
DHB	& Size	& Na\"{\i}ve    & 	Attack	& Vacc		\\
\hline
Auckland	&	436350	&	52010	&	31159	&	17920	\\
Bay of Plenty	&	206000	&	20679	&	8437	&	4585	\\
Canterbury	&	482180	&	51357	&	24695	&	13687	\\
Capital and Coast	&	283700	&	32625	&	18403	&	10461	\\
Counties Manukau	&	469300	&	55544	&	32903	&	18880	\\
Hawke's Bay	&	151700	&	15602	&	6846	&	3751	\\
Hutt Valley	&	138380	&	15198	&	7836	&	4388	\\
Lakes	&	98196	&	10558	&	5192	&	2886	\\
MidCentral	&	162560	&	17328	&	8348	&	4628	\\
Nelson Marlborough	&	137000	&	13059	&	4411	&	2356	\\
Northland	&	151690	&	14921	&	5688	&	3071	\\
South Canterbury	&	55620	&	5238	&	1678	&	893	\\
Southern	&	297420	&	31607	&	15115	&	8371	\\
Tairawhiti &	43650	&	4769	&	2431	&	1359	\\
Taranaki	&	109750	&	11473	&	5262	&	2899	\\
Waikato	&	359310	&	39402	&	20248	&	11331	\\
Wairarapa	&	41112	&	3932	&	1346	&	720	\\
Waitemata	&	525550	&	58350	&	30774	&	17291	\\
West Coast	&	32151	&	3197	&	1265	&	685	\\
Whanganui	&	60120	&	6075	&	2530	&	1378	\\
\hline			
TOTAL   & 		4241739   & 		462924   & 		234567   & 		131539 \\
\hline
\end{tabular}
\end{center}
\caption{Size: DHB Population, Statistics NZ 2013;		
Na\"{\i}ve:		DHB na\"{\i}ve population $\left(x_0\times\text{Size}\right)$;		
Attack:		Number infected in DHB in an outbreak of measles $\left(\Pe\right)$;	
Vacc:		Number to be vaccinated in DHB to reduce $\Rr_V$ below one  $\left(\left(x_0-1/\Ro\right)\times\text{Size}\right)$.			}
\label{Table1}
\end{table}%


%%%%%%%%%%%%%%%%%%%%%%%%%%%%%%%%%%%%%%%%%%%%%%%%%%%%%%%%%%%%%%%%%%%%%%%%%%%%%%%%%%%%%%%

\clearpage 
%%%%%%%%%%%%%%%%%%%%%%%%%%%%%%%%%%%%%%%%%%%%%%%%%%%%%%%%%%%%%%%%%%%%%%%%%%%%%%%%%%%%%%%
\begin{thebibliography}{11}
\bibitem{DHB} Diekmann, O., Heesterbeek, J.A.P. \& T. Britton (2013) Mathematical tools for understanding infectious disease dynamics. Princeton University Press, Princeton.
\end{thebibliography}

%%%%%%%%%%%%%%%%%%%%%%%%%%%%%%%%%%%%%%%%%%%%%%%%%%%%%%%%%%%%%%%%%%%%%%%%%%%%%%%%%%%%%%%
\end{document}