
\documentclass{article}

\usepackage{times}
\usepackage{hyperref}
\usepackage{tikz}
%\usepackage{bibunits}
\usepackage{natbib}
\usepackage{graphics}
\usepackage{amsmath}
\usepackage{indentfirst}
\usepackage[utf8]{inputenc}
\usepackage{graphicx}
\usepackage{eurosym}
\usepackage{todonotes}
\usepackage{pdflscape}
\usepackage{booktabs}
\usepackage{array}
\usepackage{rotating}
\usepackage{threeparttable}
\usepackage{blindtext}
\usepackage{dcolumn}
\usepackage{tabularx}
\usepackage{amssymb}
\usepackage{amsmath}
\usepackage{graphicx}
\usepackage{caption2}
\usepackage{fancyhdr}
\newcommand{\vecA}{\mathbf{A}}
\newcommand{\vecB}{\mathbf{B}}
\newcommand{\vecC}{\mathbf{C}}
\newcommand{\vecD}{\mathbf{D}}
\newcommand{\vecE}{\mathbf{E}}
\newcommand{\vecF}{\mathbf{F}}
\newcommand{\vecH}{\mathbf{H}}
\newcommand{\vecI}{\mathbf{I}}
\newcommand{\vecJ}{\mathbf{J}}
\newcommand{\vecK}{\mathbf{K}}
\newcommand{\vecM}{\mathbf{M}}
\newcommand{\vecN}{\mathbf{N}}
\newcommand{\vecP}{\mathbf{P}}
\newcommand{\vecQ}{\mathbf{Q}}
\newcommand{\vecS}{\mathbf{S}}
\newcommand{\vecT}{\mathbf{T}}
\newcommand{\vecW}{\mathbf{W}}
\newcommand{\vecZ}{\mathbf{Z}}
\newcommand{\vecb}{\mathbf{b}}
\newcommand{\vecc}{\mathbf{c}}
\newcommand{\vece}{\mathbf{e}}
\newcommand{\vecf}{\mathbf{f}}
\newcommand{\vecg}{\mathbf{g}}
\newcommand{\vech}{\mathbf{h}}
\newcommand{\veci}{\mathbf{i}}
\newcommand{\vecr}{\mathbf{r}}
\newcommand{\vecu}{\mathbf{u}}
\newcommand{\vecv}{\mathbf{v}}
\newcommand{\vecw}{\mathbf{w}}
\newcommand{\vecx}{\mathbf{x}}
\newcommand{\vecy}{\mathbf{y}}
\newcommand{\vecz}{\mathbf{z}}
\newcommand{\Ach}{\mathcal{H}}
\newcommand{\Ell}{\mathcal{L}}
\newcommand{\En}{\mathcal{N}}
\newcommand{\Pe}{\mathcal{P}}
\newcommand{\Ro}{\mathcal{R}_0}
\newcommand{\Rc}{\mathcal{R}_c}
\newcommand{\Rr}{\mathcal{R}}
\newcommand{\Qu}{\mathcal{Q}}
\newcommand{\Te}{\mathcal{T}}
\newcommand{\Ve}{\mathcal{V}}
\newcommand{\Dub}{\mathcal{W}}
\newcommand{\Zed}{\mathcal{Z}}
\newcommand{\Exp}{\mathbb{E}}
\newcommand{\de}{\mbox{d}}
\newcommand{\wrt}{\,\mbox{d}}
\newcommand{\veczero}{\mathbf{0}}
\newcommand{\vecone}{\mathbf{1}}
\newcommand{\vecalpha}{\mbox{\boldmath$\alpha$}}
\newcommand{\vectheta}{\mbox{\boldmath$\theta$}}
\newcommand{\vecphi}{\mbox{\boldmath$\phi$}}
\newcommand{\vecPhi}{\mbox{\boldmath$\Phi$}}
\newcommand{\vecSigma}{\mbox{\boldmath$\Sigma$}}
\newcommand{\vecimath}{\mbox{\boldmath$\imath$}}
\newcommand{\cosech}{\mbox{cosech}\,}
\newcommand{\sign}{\mbox{sign}\,}
\newcommand{\trace}{\mbox{trace}\,}
\newcommand{\Inf}{\mbox{Inf}\,}

\DeclareMathOperator{\var}{var}
\DeclareMathOperator{\cov}{cov}

\usepackage{Sweave}
\begin{document}
\Sconcordance{concordance:draftfinalreport.tex:draftfinalreport.Rnw:%
1 91 1 1 0 77 1 1 17 2 1 1 4 16 0 1 2 6 1 1 4 16 0 1 2 4 1 1 4 16 0 1 2 %
4 1 1 4 16 0 1 2 3 1 1 17 2 1 1 4 16 0 1 2 6 1 1 4 16 0 1 2 4 1 1 4 16 %
0 1 2 4 1 1 4 16 0 1 2 2 1 1 12 2 1 1 4 16 0 1 2 6 1 1 4 16 0 1 2 13 1 %
1 28 2 1 1 7 1 2 4 1 1 7 1 14 1 53 1 13 3 1 1 75 2 1 1 4 57 0 1 2 2 1 1 %
62 4 1 1 7 1 2 6 1 1 4 56 0 1 2 9 1 1 16 1 21 5 1 1 70 24 1 2 2 6 1 2 2 %
28 1 1 9 1 2 16 1 1 9 1 2 179 1 1 4 1 2 7 1 1 5 1 2 4 1 1 5 2 1 1 4 1 2 %
5 1 1 2 23 0 1 2 2 1 1 2 47 0 1 2 85 1 1 14 2 1 1 6 1 2 163 1 1 5 2 1 1 %
4 16 0 1 2 5 1 1 4 2 1 1 4 22 0 1 2 9 1 1 4 2 1 1 4 12 0 1 2 4 1 1 6 2 %
1 2 2 6 1 1 4 2 1 1 4 54 0 1 2 9 1 1 4 2 1 1 4 11 0 1 2 21 1 1 19 2 1 1 %
14 1 2 5 1 1 85 2 1 1 9 1 2 5 1 1 7 1 2 84 1 1 4 2 1 1 4 26 0 1 2 2 1 1 %
4 2 1 1 4 26 0 1 2 8 1 1 8 1 2 6 1 1 9 1 2 274 1}


\title{Interim report 2\\ Measles risk assessment, modelling and benefit--cost analysis\\ \vspace{2 mm} {\large David T S Hayman, Tim Carpenter,\\ Jonathan C Marshall, Mick Roberts, Nigel P French}}
\author{mEpiLab and EpiCentre,\\ Infectious Diseases Research Centre,\\
Massey University,\\
Palmerston North 4442,\\
New Zealand\\
\href{mailto: D.T.S.Hayman@massey.ac.nz}{D.T.S.Hayman@massey.ac.nz}}  %\texttt formats the text to a typewriter style font
\maketitle


\section{Introduction}

The well-known equation for the final size of an epidemic in a homogeneously mixing susceptible population is \citep{diekmann13}
$$\log\left(1-\Pe\right)+\Ro\Pe=0$$
where $\Ro$ is the basic reproduction number and $\Pe$ is the proportion of the population infected over the course of the outbreak.
If a proportion $x_0$ of the population is susceptible following vaccination, then the  reproduction number under vaccination is $\Rr_V=x_0\Ro$, and the final size equation becomes
$$\log\left(1-\frac{\Pe}{x_0}\right)+\Ro\Pe=0$$
Hence the relationship between the proportion initially susceptible and the proportion infected in an epidemic is
$$x_0=\frac{\Pe}{1-e^{-\Ro\Pe}}$$
In order to prevent future epidemics, it is necessary that $\Rr_V<1$. Hence, the proportion of the population that must be vaccinated to prevent future outbreaks is $x_0-1/\Ro$.

These formulae were applied at a District Health Board (DHB) level, assuming no mixing between DHBs.

\begin{table}[htdp]

\begin{center}

\begin{tabular}{lrrrr}
\hline
DHB  & Size	& Na\"{\i}ve    & 	Attack	& Vacc		\\
\hline
Auckland	&	436350	&	52010	&	31159	&	17920	\\
Bay of Plenty	&	206000	&	20679	&	8437	&	4585	\\
Canterbury	&	482180	&	51357	&	24695	&	13687	\\
Capital and Coast	&	283700	&	32625	&	18403	&	10461	\\
Counties Manukau	&	469300	&	55544	&	32903	&	18880	\\
Hawke's Bay	&	151700	&	15602	&	6846	&	3751	\\
Hutt Valley	&	138380	&	15198	&	7836	&	4388	\\
Lakes	&	98196	&	10558	&	5192	&	2886	\\
MidCentral	&	162560	&	17328	&	8348	&	4628	\\
Nelson Marlborough	&	137000	&	13059	&	4411	&	2356	\\
Northland	&	151690	&	14921	&	5688	&	3071	\\
South Canterbury	&	55620	&	5238	&	1678	&	893	\\
Southern	&	297420	&	31607	&	15115	&	8371	\\
Tairawhiti &	43650	&	4769	&	2431	&	1359	\\
Taranaki	&	109750	&	11473	&	5262	&	2899	\\
Waikato	&	359310	&	39402	&	20248	&	11331	\\
Wairarapa	&	41112	&	3932	&	1346	&	720	\\
Waitemata	&	525550	&	58350	&	30774	&	17291	\\
West Coast	&	32151	&	3197	&	1265	&	685	\\
Whanganui	&	60120	&	6075	&	2530	&	1378	\\
\hline			
TOTAL   & 		4241739   & 		462924   & 		234567   & 		131539 \\
\hline
\end{tabular}
\end{center}
\caption{Size: DHB Population, Statistics NZ 2013;		
Na\"{\i}ve:		DHB na\"{\i}ve population $\left(x_0\times\text{Size}\right)$;		
Attack:		Number infected in DHB in an outbreak of measles $\left(\Pe\right)$;	
Vacc:		Number to be vaccinated in DHB to reduce $\Rr_V$ below one  $\left(\left(x_0-1/\Ro\right)\times\text{Size}\right)$.			}
\label{Table1}
\end{table}%


\begin{thebibliography}{}
\bibliographystyle{plain}

\bibitem[Agur et al.(1993)]{agur93}
Agur, Z., L. Cojocaru, G. Mazor, R.~M. Anderson and Y.~L. Danon (1993).
\newblock Pulse mass measles vaccination across age cohorts.
\newblock \emph{Proceedings of the National Academy of Sciences USA}, 90, 11698--11702.

\bibitem[Anderson and May(1991)]{anderson91}
Anderson, R.~M. and R.~M. May (1991).
\newblock \emph{Infectious diseases of humans: dynamics and control}. Oxford: Oxford University Press.

\bibitem[Anon.(2002a)]{anon2a}
Anon. (2002a).
\newblock \emph{Immunisation handbook}
\newblock Wellington: Ministry of Health. pp.~131--146.

\bibitem[Anon.(2002b)]{anon2b}
Anon. (2002b).
\newblock \emph{Infectious diseases in livestock}
\newblock The Royal Society. pp.~68.

\bibitem[Babad et al.(1995)]{babad95}
Babad, H.~R., D.~J. Nokes, N.~J. Gay, E. Miller, P. Morgan-Capner, and R.~M. Anderson (1995).
\newblock Predicting the impact of measles vaccination in England and Wales: model validation and analysis of policy options.
\newblock \emph{Epidemiology and Infection}, 114, 319--344.

\bibitem[Bae et al.(2013)]{bae13}
Bae, G.~R, Y.~J. Choe, U.~Y. Go, Y.~I. Kim, and J.~K. Lee (2013). 
\newblock Economic analysis of measles elimination program in the Republic of Korea, 2001: A cost benefit analysis study.
\newblock \emph {Vaccine}, 31, 2661--2666.

\bibitem[Carabin et al.(2002)]{carabin2}
Carabin, H., W.~J. Edmunds, U. Kou, S. van den Hof, and V.~H. Nguyen (2002). 
\newblock Measles in industrialized countries: a review of the average costs of adverse events and measles cases.
\newblock \emph{BMC Public Health}, 2, 22.

\bibitem[Carabin et al.(2003)]{carabin3}
Carabin, H., W.~J. Edmunds, M. Gyldmark, P. Beutels, D. Levy-Bruhl, H. Salo, U.~K. and Griffiths (2003)
\newblock The cost of measles in industrialised countries.
\newblock \emph{Vaccine}, 21,4167--4177.

\bibitem[Clements and Hussey(2004]{clements4}
Clements, C.~J. and G.~D. Hussey (2004).
\newblock Chapter 4: Measles.
\newblock In \emph{The Global Epidemiology of Infectious Diseases},  Murray, C., A.~D. Lopez, and C.~D. Mathers, (eds.), Geneva.
World Health Organization, pp.~391.

\bibitem[Coleman et al.(2012)]{coleman12}
Coleman, M.~S., L. Garbat-Welch, H. Burke, M. Weinberg, K. Humbaugh, A. Tindall, and J. Cambron (2012).
\newblock Direct costs of a single case of refugee-imported measles in Kentucky.
\newblock \emph{Vaccine}, 30,317--321.

\bibitem[Dayan et al.(2005)]{dayan5}
G.~H. Dayan, I.~R. Ortega-Sanchez, C.~W. LeBaron, M.~P. Quinlisk, and the Iowa Measles Response Team (2005).
\newblock The cost of containing one case of measles: the economic impact on the public health infrastructure - Iowa, 2004.
\newblock \emph{Pediatrics}, 116:e1; DOI:10/1542/peds.2004-2512.

\bibitem[Diekmann et al.(2000)]{diekmann0}
Diekmann, O. and  J.~A.~P. Heesterbeek (2000).
\newblock \emph{Mathematical epidemiology of infectious diseases: model building, analysis and interpretation}.
Chichester: Wiley.

\bibitem[Diekmann et al.(2013)]{diekmann13}
Diekmann, O.,  J.~A.~P. Heesterbeek, and T. Britton (2013).
\newblock \emph{Mathematical tools for understanding infectious disease dynamics}.
Princeton: Princeton University Press.

\bibitem[Edmunds et al.(2000)]{edmunds0}
Edmunds, W.~J., N.~J. Gay, M. Kretzschmar, R.~G. Pebody and H. Wachman (2000).
\newblock The pre-vaccination epidemiology of measles, mumps and rubella in Europe: implications for modelling studies.
\newblock \emph{Epidemiology and Infection}, 125, 635--650.

\bibitem[Filia et al.(2007)]{filia7}
Filia, A., A. Brenna, A. Pana, G.~M. Cavallaro, M. Massari and M.~L.C. degli Atti (2007).
\newblock Health burden and economic impact of measles-related hospitalization in Italy, 2002-2003.
\newblock \emph{BMC Public Health}, 7,169

\bibitem[Flego et al.(2013)]{flego13}
Flego, K.~L., D.~A. Belshaw, V. Sheppeard, and K.~M. Weston (2013).
\newblock Impacts of a measles outbreak in western Sydney on public health resources.
\newblock \emph{Communicable Diseases Intelligence Quarterly Report}, 37, E240--245.

\bibitem[Gay et al.(1998)]{gay98}
Gay, N.~J., L. Pelletier, and P. Duclos (1998).
\newblock Modelling the incidence of measles in Canada: an assessment of the options for vaccination policy.
\newblock \emph{Vaccine}, 16, 794--801.

\bibitem[Glass et al.(2004)]{glass4}
Glass, K., J. Kappey, and B.~T. Grenfell (2004).
\newblock The effect of heterogeneity in measles vaccination population immunity.
\newblock \emph{Epidemiology and Infection}, 132, 675--683.

\bibitem[Honeycutt et al.(2006)]{honeycutt6}
Honeycutt, A. A., L. Clayton, O. Khavjou, E.~A. Finkelstein, M. Prabhu, J.~L. Blitstein, W. Dougles Evans, and J.~M. Renaud (2006).
\newblock Guide to Analyzing the Cost-Effectiveness of Community Public Health Prevention Approaches.
http://aspe.hhs.gov/health/reports/06/cphpa/report.pdf

\bibitem[Klinkenberg et al.(2011)]{klinkenberg11}
Klinkenberg, D. and H. Nishiuraa (2011).
\newblock The correlation between infectivity and incubation period of measles, estimated from households with two cases.
\newblock \emph{Journal of Theoretical Biology},284, 52--60

\bibitem[Koopmanschap(1998)]{koopmanschap98}
Koopmanschap, M.~A. (1998).
\newblock Cost-of-illness studies: useful for health policy?
\newblock \emph{Pharmacoeonomics}, 14, 143--148.

\bibitem[Larg and Moss(2011)]{larg11}
Larg, A. and J.~R. Moss (2011).
\newblock Cost-of-illness studies: a guide to critical evaluation.
\newblock \emph{Pharmacoeconomics}, 29,653--671.

\bibitem[Mansoor et al.(1998)]{mansoor98}
Mansoor, O., A. Blakely, M. Baker, M. Tobias, and A. Bloomfield (1998).
\newblock A measles epidemic controlled by immunisation. 
\newblock \emph{New Zealand Medical Journal}, 111, 467--471.

\bibitem[Ortega-Sanchez et al.(2014)]{ortegasanchez14}
Ortega-Sanchez, I.~R., M. Vijayaraghavan, A.~E. Barskey, and G.~S. Wallace (2014).
\newblock The economic burden of sixteen measles outbreaks on United States public health departments in 2011.
\newblock \emph{Vaccine}, 32,1311--1317.

\bibitem[Obidia et al.(2012)]{obidia12}
Obadia, T., R. Haneef and P--Y. Boelle
\newblock The R0 package: a toolbox to estimate reproduction numbers for epidemic outbreaks.
\newblock \emph{BMC Medical Informatics and Decision Making}, 2012, 12--147.

\bibitem[Parker et al.(2006)]{parker6}
Parker, A.~A., W. Staggs, G.~H. Dayan, I.~R. Ortega-Sanchez, P.~A. Rota, L. Lowe, P. Boardman, R. Teclaw, C. Graves, and C.~W. LeBaron (2006).
\newblock Implications of a 2005 measles outbreak in Indiana for sustained elimination of measles in the United States.
\newblock \emph{The New England Journal of Medicine}, 355, 447--455.

\bibitem[Prouty et al.(2001)]{prouty1}
Prouty, R.W., G. Smith and K.~C. Lakin (2001).
\newblock Residential services for persons with developmental disabilities: status and trends through 2000.
\newblock \emph{Minneapolis: Institute on Community Integration}, University of Minnesota, pp.~179, rtc.umn.edu/risp00.

\bibitem[Roberts(2004)]{roberts4}
Roberts, M. (2004).
\newblock A mathematical model for measles vaccination.
\newblock Wellington: Ministry of Health.

\bibitem[Roberts and Tobias(2000)]{roberts0}
Roberts, M.~G. and M.~I. Tobias (2000).
\newblock Predicting and preventing measles epidemics in New Zealand: Application of a mathematical model. 
\newblock \emph{Epidemiology and Infection}, 124, 279--287.

\bibitem[Saha and Gerdtham(2013)]{saha13}
Saha, S. and U.~G. Gerdtham (2013).
\newblock Cost of illness studies on reproductive, maternal, newborn, and child health: a systematic literature review.
\newblock \emph{Health Economics Review}, doi:10.1186/2191-1991-3-24.

\bibitem[Siedler et al.(2006)]{siedler6}
Siedler, A., A. Tischer, A. Mankertz, and S. Santibanez (2006).
\newblock Two outbreaks of measles in Germany 2005.
\newblock \emph{Eurosurveillance} 2006:11(4) article 5, \href{http://www.eurosurveillance.org/ViewArticle.aspx?ArticleId=615}{www.eurosurveillance.org}, accessed 14 June 2014.

\bibitem[Stack et al.(2011)]{stack11}
Stack, M.~L., S. Ozawa, D.~M. Bishai, A. Mirelman, Y. Tam, L. Niessen, D.~G. Walker, and O.S. Levine (2011).
\newblock Estimated economic benefits during the 'decade of vaccine' include treatment savings, gains in labor productivity.
\newblock \emph{Health Affairs}, 30,1021--1028.

\bibitem[Statistics New Zealand(2014))]{stats14}
\newblock \emph{Statistics New Zealand} (2014).
http://nzdotstat.stats.govt.nz/, accessed 17 June 2014.

\bibitem[Tobias and Roberts(1998)]{tobias98}
Tobias, M.~I. and M.~G. Roberts (1998).
\newblock Predicting and preventing measles epidemics in New Zealand: Application of a mathematical model.
\newblock Wellington: Ministry of Health.

\bibitem[Wallinga et al.(2001)]{wallinga1}
Wallinga, J., D. Levy-Bruhl, N.~J. Gay, and C.~H. Wachman (2001).
\newblock Estimation of measles reproduction ratios and prospects for elimination of measles by vaccination in some Western European countries.
\newblock \emph{Epidemiology and Infection}, 127, 281--295.

\bibitem[Wallinga and Teunis(2004)]{wallinga4}
Wallinga, J., and P. Teunis (2004).
\newblock Different Epidemic Curves for Severe Acute Respiratory Syndrome Reveal Similar Impacts of Control Measures.
\newblock \emph{American Journal of Epidemiology}, 160, 509.

\bibitem[Wichmann et al.(2009)]{wichmann9}
Wichmann, O., A. Siedler, D. Sagebiel, W. Hellenbrand, S. Santibanez, A. Mankertz, G. Vogt, U. van Treeck, and G. Krause (2009).
\newblock Further efforts needed to achieve measles elimination in Germany: results of an outbreak investigation.
\newblock \emph{Bulletin of the World Health Organization}, 87, 108--115.

\bibitem[Wolfson et al.(2007)]{wolfson7}
Wolfson, L.~J., P.~M. Strebel, M. Gacic-Dobo, E.~J. Hoekstra, J.~W. McFarland, and B.~S. Hersh (2007).
\newblock Has the 2005 measles mortality reduction goal been achieved? A natural history modelling study.
\newblock \emph{Lancet}, 369, 191--200.

\bibitem[WHO(2013)]{who13}
World Health Organisation measles media centre, January (2013)
\newblock Geneva: World Health Organization.
\href{http://www.who.int/mediacentre/news/notes/2013/measles_20130117/en/}{www.who.int}, accessed July 1, 2014.

\bibitem[Zhou et al.(2004)]{zhou4}
Zhou, F, S. Reef, M. Massoudi, M.~J. Papania, H.~R. Yusuf, B. Bardenheier, L. Zimmerman, and M.~M. McCauley (2004).
\newblock An economic analysis of the current universal 2-dose measles-mumps-rubella vaccination program in the United States.
\newblock \emph{Journal of Infectious Diseases}, 189, S131--45.

\end{thebibliography}

\end{document}
